\chapter{ Marco Teórico }

\section{ Arquitectura SOA }
SOA es una arquitectura de software distribuida orientada a servicios, en esta toda la lógica de negocio del sistema se divide en unidades elementales llamadas servicios, los cuales representan una parte del flujo de información en un sistema.
Los servicios pueden tener diferentes implementaciones, pero la más común y eficiente son los Servicios Web (Web Services).

\section{ Servicios Web }
Un servicio web es un programa independiente que representa un proceso dentro de la lógica de negocio de un sistema con interfaces abiertas utilizando protocolos de Internet.

La descripción de un servicio web involucra los siguientes elementos:

\begin{itemize}
\item Lenguaje base en común. Se debe considerar la adopción de un lenguaje base para transmitir los datos que el servicio web exponga o reciba, durante años el estándar más utilizado era XML, pero JSON ha sobresalido en años recientes por ser un formato más ligero y fácil de entender, lo que mejora el rendimiento y universalidad de los servicios.
\item  Interfaces. Es necesario definir de manera correcta la interfaz de los servicios web, pues se debe indicar el URI y el protocolo con el que se obtiene acceso al servicio, los datos que requiere y los datos que regresan.
\item Protocolos de Negocio. En ocasiones para completar una operación requiere llamar a varios servicios para ser completada, por lo que es importante definir qué servicios y en qué orden se llaman.
\item Propiedades. Las propiedades pueden representar características no funcionales de los servicios, por ejemplo una descripción textual del servicio o la versión del mismo.
\end{itemize}


Para definir la estructura de los servicios web se deben considerar dos aspectos, primero que son una forma de exponer funcionamiento interno de un sistema a clientes externos, y segundo que son aplicaciones independientes y llevan a cabo procesos internos.
\begin{itemize}
\item 
\end{itemize}

\subsection{ Estructura interna de un servicio Web }
La estructura interna de un servicio web puede definirse en 4 aspectos clave 


\subsection{ Estructura externa de un Servicio Web }



\section{ Java }
\section{ Java EE }
\section{ Spring }
Spring es un framework para para la construcción aplicaciones web desarrollado sobre Java EE. Es una de las implementaciones web de Java más utilizadas en la industria, por lo tanto, una de las más estables y eficientes. Spring permite  
\section{json}
\section{PostgreSQL}
\section{ORM}
\section{Hibernate}
\section{Java Server Faces}
\section{PrimeFaces}
\section{WildFly}